\documentclass{beamer}
\usepackage{amsmath}
\usepackage{amsfonts}
\usepackage{amssymb}
\usepackage{tikz}
\usepackage{multicol}
\usepackage[spanish]{babel}
\usepackage{pgfgantt}
\usepackage{xcolor}
\usetheme{Warsaw}
\title[Coordinaci\'on Acad\'emica ]{Presentaci\'on Resumen Actividad 2014}
\author{Pamela Delgado Carrasquilla y Sebasti\'an Rosales Cote }
\begin{document}
\begin{frame}
\maketitle
\end{frame}
\begin{frame}
\tableofcontents
\end{frame}
\section{Estructura}
\begin{frame}
\frametitle{El equipo}
\begin{tikzpicture}
\node[circle, draw=black, fill=blue!60](pame) at (0,1.3) {Pame}; 
\node[circle, draw=black, fill=blue!40](sebas) at (0,0) {Sebas};
	\draw[<->]  (pame) to (sebas);
\node[circle, draw=black, fill=blue!30](lau) at (0,-2.5) {Lau M};
\node[circle, draw=black, fill=blue!30](fede) at (-2.5,-0){Fede O};
	\draw[<->]  (fede) to (sebas);
	\draw[<->]  (lau) to (sebas);
\node[circle, draw=black, fill=blue!30](andre) at (2.5, -0) {Andre C};
	\draw[<->]  (andre) to (sebas);
\node[circle, draw=black, fill=blue!30](hermosa) at (-1.76, -1.76) {Andre R};
	\draw[<->]  (hermosa) to (sebas);
\node[circle, draw=black, fill=blue!30](cata) at (1.76, -1.76) {Cata E};
	\draw[<->]  (cata) to (sebas);
\node[circle, draw=black, fill=blue!30](jose) at (0, -5) {Jose T};

	
	
\node (juli) at (3.5, 1.2){Juli};
	\draw [<->] (andre) to( juli);
\node (aleja) at (4, 0.4){Aleja};
	\draw [<->] (andre) to( aleja);
\node (luis) at (4, -0.4){Luis};
	\draw [<->] (andre) to( luis);
\node (oscar) at (3.5, -1.2){Oscar};
	\draw [<->] (andre) to( oscar);

\node (nata) at (-3.5, 1.2){Nathalia};
	\draw [<->] (fede) to( nata);
\node (manu) at (-4, 0.4){Manuela};
	\draw [<->] (fede) to( manu);
\node (laura) at (-4, -0.4){Laura};
	\draw [<->] (fede) to( laura);
\node (andrear) at (-3.5, -1.2){Andre};
	\draw [<->] (fede) to( andrear);

\node (orduz) at (-2.75, -2.9){Orduz};
	\draw [<->] (hermosa) to( orduz);
\node (lore) at (-3.15, -2.5){Lore};
	\draw [<->] (hermosa) to( lore);
\node (q) at (-3.5, -2){Qui\~nonez};
	\draw[<->] (q) to (hermosa);
\node (dani) at (-1.75, -3.1){Dani};
	\draw[<->] (dani) to (hermosa);
	
\node (orduz) at (2.75, -2.9){Andre G};
	\draw [<->] (cata) to( orduz);
\node (lore) at (3.15, -2.5){Angela};
	\draw [<->] (cata) to( lore);
\node (q) at (3.5, -2){Aleja C};
	\draw[<->] (q) to (cata);
\node (dani) at (1.75, -3.1){Diana};
	\draw[<->] (dani) to (cata);

\node(bayron) at (0,-4) {Bayron};
	\draw [<->] (lau) to (bayron);
\end{tikzpicture}
\end{frame}
\begin{frame}
\frametitle{Las Funciones}
\begin{center}
\begin{tikzpicture}
  \node[ circle,draw=black!50,fill=blue!40 ] (pl) at (3.5 ,-2.3){ Planeaci\'on  };
  \node[circle,draw=black!50,fill=blue!40] (ej) at (-3.5 ,-2.3 ){ Ejecuci\'on };
  \node[circle,draw=black!50,fill=blue!40] (ev) at (0,3.2 ){Evaluaci\'on };
     \draw[->] (pl) to [out=210,in=330](ej);
     \draw[->] (ej) to [out=90,in=210](ev);
     \draw[->] (ev) to [out=330,in=90](pl);
  \node[rectangle,draw=black!50,fill=orange!25!green] (c) at (2,0 ){Curriculum };
   \node[rectangle,draw=black!50,fill=orange!25!blue] (d) at (-2,0 ){Docentes };
  \node[rectangle,draw=black!50,fill=orange!25!red] (r) at (0,1.5 ){Recursos };
  \node[rectangle,draw=black!50,fill=orange!25!yellow] (s) at (0,-1.5 ){Estudiantes };
   	 \draw[<->] (r) to [out=270,in=90](s);
     \draw[<->] (d) to [out=0,in=180](c);
     \draw[<->] (c) to (r);
     \draw[<->] (r) to (d);
     \draw[<->] (d) to (s);
     \draw[<->] (s) to (c);
 
\end{tikzpicture}
\end{center}
\end{frame}
\section{Motivaciones y Marcos}
\begin{frame}
\frametitle{Marcos}
\begin{block}{Marco Pedag\'ogico}
Orientaci\'on cr\'itico progresista, teniendo en cuenta que se parte de un sistema positivista y las limitaciones de tiempo, docentes y recursos. 
\end{block}
\begin{block}{ Marco Legal}
Lineamientos y disposiciones del MEN para educaci \'on formal. 
\end{block}
\begin{block}{Poblaci\'on}
Poblaci\'on adulta con presaberes y habilidades desarrolladas durante la vida. Con necesidad de herramientas practicas y de desarrollo de habilidades mentales. 
\end{block}
\end{frame}


\begin{frame}
\frametitle{Motivaciones}
\begin{enumerate}
\item Tiempo insuficiente para dictar las materias. 
\item Contenidos no pertinentes para los estudiantes. 
\item Desperdicio del tama\~no de los grupos.
\item ?`Est\'an aprendiendo los estudiantes?
\item Met\'odologias didacticas no motivantes.
\item Evaluaci\'on no adecuada. 
\end{enumerate}
\end{frame}
 \begin{frame}
\begin{center}
\begin{tikzpicture}
	\filldraw[draw=black, fill=orange!25!green!25] (0,0) circle (3.5);
	 \node[rectangle,draw=black,fill=green] (mc) at (3,3 ){Marco Fil\'osofico};
  \node[rectangle,draw=black,fill=green] (ml) at (-3,3){Marco Legislativo};
  \node[rectangle,draw=black,fill=green] (mr) at (3,-3 ){Marco Referencial };
   \node[rectangle,draw=black,fill=green] (mp) at (-3,-3){Marco Pedag\'ogico};
   
  \node[rectangle,draw=black!50,fill=orange!25!green] (c) at (2,0 ){Objetivos };
  \node[rectangle,draw=black!50,fill=orange!25!green] (d) at (-2,0 ){Contenidos };
  \node[rectangle,draw=black!50,fill=orange!25!green] (r) at (0,1.5 ){M\'etodos };
   \node[rectangle,draw=black!50,fill=orange!25!green] (s) at (0,-1.5 ){Sistema de  Evaluaci\'on};
   \draw[<->] (r) to [out=270,in=90](s);
     \draw[<->] (d) to [out=0,in=180](c);
     \draw[<->] (c) to (r);
      \draw[<->] (r) to (d);
      \draw[<->] (d) to (s);
      \draw[<->] (s) to (c);
\end{tikzpicture}
\end{center}
\end{frame}


\section{Actividades Realizadas }
\begin{frame}
\frametitle{Objetivos}
\begin{center}
\begin{tikzpicture}
  \node[rectangle,draw=black!50,fill=orange!25!green] (pe) at (0,4 ){Perfil del egresado };
  \node[rectangle,draw=black!50,fill=orange!25!green!50] (ps) at (0,2 ){Prop.  Sociales};
    \draw[->] (pe) to [out=270,in=90](ps);
  \node[rectangle,draw=black!50,fill=orange!25!green!50] (pm) at (-2.25,1 ){Prop. Mate. };
    \draw[->] (pe) to [out=270,in=90](pm);
   \node[rectangle,draw=black!50,fill=orange!25!green!50] (pes) at (2.25,1 ){Prop. Espa\~nol };
    \draw[->] (pe) to [out=270,in=90](pes);
   \node[rectangle,draw=black!50,fill=orange!25!green!50] (pi) at (4.,2 ){Prop. Ingl\'es };
    \draw[->] (pe) to [out=270,in=90](pi);
   \node[rectangle,draw=black!50,fill=orange!25!green!50] (pn) at (-4,2 ){Prop. Naturales  };
    \draw[->] (pe) to [out=270,in=90](pn);

\end{tikzpicture}
\end{center}
\end{frame}

\begin{frame}
\frametitle{Perfil del Egresado}
\begin{enumerate}
\item \textbf{Conocimientos:}
\begin{enumerate}[a]
\item Familia y Comunidad.
\item Desempe\~no autentico.
\item Valores.
\end{enumerate}
\item \textbf{Habilidades}: 
\begin{enumerate}[a]
\item Servicio
\item Liderazgo
\end{enumerate}
\item \textbf{Capacidades}:
\begin{enumerate}[a]
\item Desempe\~no aut\'entico
\item Entender y mejorar su entorno y a s\'i mismo.
\item Valor intr\'inseco de las personas. 
\end{enumerate}
\end{enumerate}
\end{frame}

\begin{frame}
\frametitle{Esp\'iritu de los propositos}
\begin{itemize}
\item \textbf{Matem\'aticas:} Abstracci\'on y pensamiento l\'ogico $\rightarrow$ Resolver problemas ec\'onomico-financieros y laborales. 
\item \textbf{Naturales: } Curiosidad,creatividad, abstracci\'on y cuestionamiento $\rightarrow$  Visualizarse como agente constructores de conocimiento y mejorar su calidad de vida. 
\item \textbf{Sociales:} Pensamiento cr\'itico $\rightarrow$ Asumirse como ciudadadano y agente constructor de conocimiento y de transformaci\'on social. 
\item \textbf{ Espa\~nol:} Sujeto: $\rightarrow $  Lenguaje y sus manifestaciones. $\leftrightarrow$ Configuraci\'on de Identidad $\leftrightarrow$  Interacci\'on social $\leftarrow$
\item \textbf{Ingl\'es: } Analizar el lenguaje como medio para acercarse a la realidad $\rightarrow$  Fomentar la curiosidad por diferentes culturas. 
\end{itemize}
\end{frame}
\begin{frame}
\frametitle{De objetivos a contenidos}
\begin{center}
\begin{tikzpicture}
\node[rectangle,draw=black!50,fill=orange!25!green!50] (pe) at (0,4 ){Prop\'osito Sociales};
\node[rectangle,draw=black!50,fill=orange!25!green!25] (ps) at (0,2 ){Objetivo-Contenido};
    \draw[->] (pe) to [out=270,in=90](ps);
  \node[rectangle,draw=black!50,fill=orange!25!green!25] (pm) at (-2.25,1 ){Objetivo-Contenido };
    \draw[->] (pe) to [out=270,in=90](pm);
   \node[rectangle,draw=black!50,fill=orange!25!green!25] (pes) at (2.25,1 ){Objetivo-Contenido };
    \draw[->] (pe) to [out=270,in=90](pes);
   \node[rectangle,draw=black!50,fill=orange!25!green!25] (pi) at (4.,2 ){Objetivo-Contenido };
    \draw[->] (pe) to [out=270,in=90](pi);
   \node[rectangle,draw=black!50,fill=orange!25!green!25] (pn) at (-4,2 ){Objetivo-Contenido};
    \draw[->] (pe) to [out=270,in=90](pn);

\end{tikzpicture}
\end{center}
\end{frame}
\begin{frame}
\fontsize{9pt}{10}\selectfont
\frametitle{Plan de Estudios}
\begin{table}
\centering
\begin{tabular}{|c|p{2cm}|p{2cm}|p{2cm}|p{2cm}|}
\hline 
\textbf{Curso} & \textbf{Matem\'aticas} &\textbf{Sociales} &\textbf{Espa\~nol} & \textbf{Ciencias}\\
\hline
\underline{Cuarto} & Operaciones con naturales y Resoluci\'on de problemas   & Hablando del Universo y del espacio.  & Oralidad y la Palabra & Ecolog\'ia, diversidad, contaminaci\'on y reciclaje.  \\
\underline{Quinto} & Operaciones con enteros.  & La historia. Contruyendo de comunidad. & La escritura &Conceptos b\'asicos. Circulaci\'on y locomoci\'on.  \\
\underline{Sexto} & Fraccionarios y proporciones & El gran relato de Occidente. \'America y el renacimiento  & La lectura (Contenido)  & Metabolismo y sistema nervioso \\
\hline
\end{tabular}
\end{table}
\end{frame}


\begin{frame}
\fontsize{9pt}{10}\selectfont
\frametitle{Plan de Estudios}
\begin{table}
\centering
\begin{tabular}{|c|p{2cm}|p{2cm}|p{2cm}|p{2cm}|}
\hline 
\textbf{Curso} & \textbf{Matem\'aticas} &\textbf{Sociales} &\textbf{Espa\~nol} & \textbf{Ciencias}\\
\hline
\underline{Septimo} & Geometr\'ia & ??? & Formas de utilizaci\'on de lenguaje & Reproducci\'on \\
\underline{Octavo} & Principios tras el algebra (Variables, igualdades)  &El otro y el siglo XX & Contexto Comunicativo  & Evoluci\'on, Filogen\'etica y  alimentaci\'on \\
\underline{Noveno} & Algebra & Colombia y colombianos: Contruyendo historia. (Seminario - Taller)  & Medios de Comunicaci\'on y otros  sistemas simbolicos & Qu\'imica \\
\hline 
\end{tabular}
\end{table}
\end{frame}
\begin{frame}
\fontsize{9pt}{10}\selectfont
\frametitle{Plan de Estudios}
\begin{table}
\centering
\begin{tabular}{|c|p{2cm}|p{2cm}|p{2cm}|p{2cm}|}
\hline 
\textbf{Curso} & \textbf{Matem\'aticas} &\textbf{Sociales} &\textbf{Espa\~nol} & \textbf{Ciencias}\\
\hline
\underline{Decimo} & Funciones (Cuadr\'aticas y trigonometr\'ia)  & Filosof\'ia: Conocimiento, Moral y Justicia & Pensamiento Cr\'itico & D\'inamica y Termodin\'anica\\
\underline{Once} & Probabilidad y Estad\'istica & Yo, ciudadano. & Literatura & Electromagn. , Luz y Sonido \\
\hline 
\end{tabular}
\end{table}
\end{frame}

\begin{frame}
\fontsize{9pt}{10}\selectfont
\frametitle{Plan de Estudios. Ingles.}
\begin{tikzpicture}
\node (pronuons ) at (0,2) {Pronouns};
\node (tb ) at (-1,0.27) {To be};
\node( th) at ( -1.73, -0.5 ) {Determinantes};
\end{tikzpicture}
\end{frame}
\section{2015}
\end{document}